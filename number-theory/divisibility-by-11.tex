\documentclass{article}
\usepackage{amsthm,amssymb,amsmath}
\usepackage[shortlabels]{enumitem}
\usepackage[utf8]{inputenc}
\usepackage[english]{babel}

\newtheorem*{problem*}{Problem}
\newtheorem*{solution*}{Solution}
\renewcommand{\theenumi}{\alph{enumi}\)}
\newcommand{\Mod}[1]{\ (\mathrm{mod}\ #1)}

\title{Divisibility by 11}
\date{6/23/2020}

\begin{document}
\maketitle
 
\begin{problem*}
    How to show that a number is divisible by $11$?
\end{problem*}

\begin{solution*}
    Let's look at division. When a number $a$ is divided by another
    number $x$ and leaves $b$ as remainder, we can write
    \begin{equation}
        a = n \cdot x + b
    \end{equation}
    
    where $n$ is called the "quotient". But if we just focus on the 
    remainder, we call this relationship \textbf{modular equivalence}
    and write it as $a \equiv b \Mod x$. Now consider another such statement,
    \begin{equation}
        c = m \cdot x + d
    \end{equation}

    or in terms of \textbf{modular equivalence},
    $c \equiv d \Mod x$\\

    On adding both sides of $(1)$ and $(2)$, we get 
    $(a+c) = (n+m) \cdot x + (b+d)$. It looks like
    when the number $(a+c)$ is divided by $x$, it leaves
    $(b+d)$ as the remainder. This property can be written as
    \begin{equation}
        (a+c) \equiv (b+d) \Mod x
    \end{equation}

    If we multiplied both sides of $(1)$ by a constant number, $k$
    we would see $ak = nk \cdot x + bk$ and again, focussing just on the remainder:
    \begin{equation}
        ak \equiv bk \Mod x
    \end{equation}

    We shall use $(3)$ and $(4)$ as \textbf{properties of modular equivalence}.\\

    We see that $10 = 1 \cdot 11 + (-1)$ which, in terms of
    \textbf{modular equivalence} we be written as 
    $10 \equiv -1 \Mod {11}$.\\

    Now let's consider all the non-negative powers of $10$
    with respect to divisibility by $11$. We will see that
    \begin{align*}
        1 &\equiv 1 \Mod {11}\\
        10 &\equiv -1 \Mod {11}\\
        100 &\equiv 1 \Mod {11}\\
        1000 &\equiv -1 \Mod {11}\\
        10000 &\equiv 1 \Mod {11}\\
        \vdots\\
    \end{align*}

    Multiplying all these modular equivalences by arbitrary
    non-negative constants $a_0, a_1, \dots a_n$ and using
    \textbf{property $(4)$} above:
    \begin{align*}
        1\cdot{a_0} &\equiv a_0 \Mod 9\\
        10\cdot{a_1} &\equiv -a_1 \Mod 9\\
        100\cdot{a_2} &\equiv a_2 \Mod 9\\
        1000\cdot{a_3} &\equiv -a_3 \Mod 9\\
        10000\cdot{a_4} &\equiv a_4 \Mod 9\\
        \vdots\\
        10^n\cdot{a_n} &\equiv \pm a_n \Mod 9\\
    \end{align*}

    Adding all these equivalences using \textbf{property $(3)$}:
    \begin{equation*}
        1\cdot{a_0} + 10\cdot{a_1} + 100\cdot{a_2} + 1000\cdot{a_3} + 10000\cdot{a_4} \dots + 10^n\cdot{a_n} 
        \equiv
        (a_0 - a_1 + a_2 - a_3 + a_4 \dots + a_n) \Mod {11}\\
    \end{equation*}

    Well! the left hand side of the above equivalence is nothing
    but a representation of a $(n+1)$-digit number 
    (e.g., $76934 = 
    1\cdot{4} + 10\cdot{3} + 100\cdot{9} + 1000\cdot{6} + 10000\cdot{7}$)
    and the right hand side is the alternating sum and difference
    of its digits!\\

    Hence the equivalence statement above shows that 
    \textbf{when dividing a number by $11$ its remainder 
    is the same as that when dividing the alternating difference ans sum
    of its digits (starting with a difference)}.\\

    E.g., $76934$ is divisible by $11$
    since the sum of its digits is $11$
\end{solution*}

\end{document}
