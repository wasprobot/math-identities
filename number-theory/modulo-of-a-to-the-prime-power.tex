\documentclass{article}
\usepackage{amsthm,amssymb,amsmath}
\usepackage[shortlabels]{enumitem}
\usepackage[utf8]{inputenc}
\usepackage[english]{babel}

\newtheorem*{problem*}{Problem}
\newtheorem*{solution*}{Solution}
\renewcommand{\theenumi}{\alph{enumi}\)}

\title{Prime power of an integer modulo the prime}
\date{11/11/2020}

\begin{document}
\maketitle
 
\begin{problem*}
    To prove that for a prime $n$, every 
    integer smaller than $n$
    raised to the power $n$ is equivalent to itself
    modulo $n$ ($a^n \equiv a\mod{n}$).
    In other words $a^n$ when divided by $n$
    leaves $a$ as remainder.
\end{problem*}

\begin{solution*}
    We prove this by induction. Using $a=1$ for the 
    \textbf{basis step}, we can see that $1^n \equiv 1\mod{n}$
    (i.e., when divided by $n$, $1$ leaves the remainer $1$).\\

    For the \textbf{induction step} assume the claim is 
    true for an arbitrary $a=k<n$. I.e., $k^n \equiv k\mod{n}$, 
    or for a quotient, $q$:
    \begin{equation}
        k^n = n\cdot{q} + k
    \end{equation}

    Now consider, for $k+1<n$, the \textbf{binomial expansion}:
    \begin{align*}
        (k+1)^n
        &= k^n + \sum_{r=1}^{n}{n\choose{r}}k^{n-r} + 1\\
        &= (n\cdot{q} + k) + \sum_{r=1}^{n}{n\choose{r}}k^{n-r} + 1
    \end{align*}
    \begin{equation}
        (k+1)^n = n\cdot{q} + \sum_{r=1}^{n}{n\choose{r}}k^{n-r} + (k+1)
    \end{equation}
    
    The binomial coefficient, $n\choose{r}$
    also expressed as $\frac{n!}{j}$ is an integer.
    However $n$, being prime, can be extracted from 
    it leaving behind an integer $i_r$ ($j$ does not divide $n$).
    Thus (2) can be re-written as:
    \begin{align*}
        (k+1)^n
        &= n \left(q + \sum_{r=1}^{n}{i_r}k^{n-r} \right) + (k+1)\\
        &= n\cdot{p} + (k+1) & \text{for an integer $p$}
    \end{align*}

    But since $k+1<n$, $(k+1)^n \equiv (k+1)\mod{n}$. 
    In other words, $(k+1)^n$ when divided by $n$
    leaves $k+1$ as remainder. Which means the claim is
    true for $a=(k+1)<n$. And since we got to this conclusion
    by assuming the claim were true for $a=k<n$ 
    it must be true for all $a<n$.
\end{solution*}
\end{document}