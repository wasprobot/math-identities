\documentclass{article}
\usepackage{amsthm,amssymb,amsmath}
\usepackage[shortlabels]{enumitem}
\usepackage[utf8]{inputenc}
\usepackage[english]{babel}

\newtheorem*{problem*}{Problem}
\newtheorem*{solution*}{Solution}
\renewcommand{\theenumi}{\alph{enumi}\)}

\title{Prime power of an integer modulo the prime}
\date{11/11/2020}

\begin{document}
\maketitle
 
\begin{problem*}
    To prove that for a prime, $n$ every 
    integer smaller than $n$,
    raised to the power $n$ is equivalent to itself
    modulo $n$ ($a^n \equiv a\mod{n}$).
    In other words $a^n$ when divided by $n$
    leaves $a$ as remainder.
\end{problem*}

\begin{solution*}
    We prove this by induction. Using $a=1$ for the 
    basis step, we can see that $1^n \equiv 1\mod{n}$
    (i.e., when divided by $n$, $1$ leaves the remainer $1$).\\

    Now let's assume the claim is true for an arbitrary
    $a=k$: I.e., $k^n \equiv k\mod{n}$, or for a quotient,
    $q$
    \begin{equation}
        k^n = n\cdot{q} + k
    \end{equation}

    Now consider the \textbf{binomial expansion}:
    \begin{align*}
        (k+1)^n
        &= k^n + \sum_{r=1}^{n}{n\choose{r}}k^{n-r} + 1\\
        &= n\cdot{q} + \sum_{r=1}^{n}{n\choose{r}}k^{n-r} + (k+1)
        &\text{using (1)}
    \end{align*}
    The binomial coefficient, $n\choose{r}$ is an integer.
    However since $n$, being prime, can be extracted from 
    it still leaving an integer $i_r$. Therefore
    \begin{align*}
        (k+1)^n
        &= n\cdot{q} + \sum_{r=1}^{n}n{i_r}k^{n-r} + (k+1)\\
        &= n\cdot{q} + n \sum_{r=1}^{n}{i_r}k^{n-r} + (k+1)\\
        &= n\cdot{p} + (k+1)
        & \text{for an integer $p$}\\
        \therefore (k+1)^n &\equiv (k+1)\mod{n}
    \end{align*}
    Thus we see that assuming the claim is 
    true for $a=k$ we prove that it is true
    for $a=k+1$. Hence it's true for all $a<n$
\end{solution*}
\end{document}