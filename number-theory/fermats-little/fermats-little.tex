\documentclass{article}
\usepackage[utf8]{inputenc}
\usepackage[english]{babel}
\usepackage{amsmath,amssymb}
\usepackage{mathtools}
\usepackage{amsfonts}
\usepackage{amssymb}
\usepackage{tikz}
\usepackage{amsthm}

\newtheorem*{conjecture}{Conjecture}
\newtheorem*{corollary}{Corollary}
\newtheorem*{proposition}{Proposition}

\title{Fermat's Little Theorem}

\begin{document}

\maketitle

\begin{conjecture}
If $a,p \in \mathbb{N}$, $p$ is prime and $gcd(a,p) = 1$
then $a^{p-1} \equiv 1 \pmod p$.
\end{conjecture}

\begin{proof}
We claim that the first $(p-1)$ multiples of $a=\{a,2a,3a,\dots,(p-1)a\}$, when divided by $p$, have distinct, non-zero remainders. Let $\mathbb{Z}_{p-1}$ represent the set of first $(p-1)$ positive integers. Let $k \in \mathbb{Z}_{p-1}$. If $ka$ had a zero remainder on division by $p$, it would mean $p \mid ka$.\\

We will prove that this means $p \mid k$ or $p \mid a$.
Since $p\mid ka$ there must exist $x \in \mathbb{Z}$ so that $px=ka$. Assume $p \nmid k$.

\end{proof}

% \begin{corollary}
%     If $p \in \mathbb{N}$ is prime and $p \ge 7$
%     then $p$ has a multiple whose digits are $p-1$ $1$'s.
% \end{corollary}
    
% We define \textit{Repunit}, a natural number
whose digits are all $1$'s (e.g., $1, 11, 111 \dots$).
$r_n$ is a repunit with $n$ digits:

\begin{equation*}
r_n = 10^0 + 10^1 + 10^2 \dots + 10^{n-1}
= \sum_{k=0}^{n-1}{10^k}
\end{equation*}
    
\end{document}
