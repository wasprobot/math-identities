\documentclass{article}
\usepackage{amsthm,amssymb,amsmath,multirow,multicol}
\usepackage[shortlabels]{enumitem}
\usepackage[utf8]{inputenc}
\usepackage[english]{babel}

\newtheorem*{problem*}{Problem}
\newtheorem*{solution*}{Solution}
\renewcommand{\theenumi}{\alph{enumi}\)}

\title{Dougy's question on money}
\date{7/16/2025}

\begin{document}
\maketitle

\begin{problem*}
    Doug asked a very interesting question: \textbf{How many coins 
    do we need to carry if we were to be able to pay 
    any denomination from 0 to 99 cents?}\\

    I took the liberty to change his question into something 
    completely different: \textbf{Given we have enough coins of each kind
    (Quarters, Dimes, Nickles, and Pennies) in how many different ways 
    can we represent money from 1 to 100 cents?}\\

    So here we will examine my question, and ignore his.
\end{problem*}

\begin{solution*}
    We introduce a function, $C(100, Q, D, N, P)$,
    representing the number of such combinations,
    using quarters (Q), dimes (D), nickles (N), and pennies (P)\\

    We make some simple observations. One is that $C(0,\cdots)=1$: 
    There is one combination to pay 0 cents (that is to use no coins!)\\
    
    Also $C(100,Q)=1$ (use 4 quarters), and so on\dots\\

    As we begin to dive deeper, we see that $C(5, N, P)=2$, the
    two combinations that can be visualized with the table:\\

    \begin{tabular}{|c|c|}
        \hline
        N & P \\
        \hline
        0 & 5\\
        \hline
        1 & 0\\
        \hline
    \end{tabular}\\\\

    And what if we were to represent 100 cents, 
    using only Nickles and Pennies? $C(100, N, P)=21$, because:\\

    \begin{tabular}{|c|c|}
        \hline
        N & P \\
        \hline
        0 & 100\\
        \hline
        1 & 95\\
        \hline
        2 & 90\\
        \hline
        \dots & \dots\\
        \hline
        20 & 0\\
        \hline
    \end{tabular}\\\\  

    Similarly, $C(100, D, N, P)=$, because:\\

    \begin{tabular}{|n|p|s}
        \hline
        N & P \\
        \hline
        0 & 100\\
        \hline
        1 & 95\\
        \hline
        2 & 90\\
        \hline
        \dots & \dots\\
        \hline
        20 & 0\\
        \hline
    \end{tabular}\\\\  

    \begin{tabular}{|c|c|c|}
        \hline
        D & N & P \\
        \hline
        0 & \multicolumn{2}{|c|}{
            \multirow{3}{*}{
                \begin{tabular}{|l|r|}
                    \hline
                    Sub-A & 10 \\
                    \hline
                    Sub-B & 20 \\
                    \hline
                \end{tabular}%
            }}%
        } \\
        \cline{1-1} % Line only for Column 1
        0 & \multicolumn{2}{|c|}{} \\
        \cline{1-1} % Line only for Column 1
        0 & \multicolumn{2}{|c|}{} \\
        \hline
    \end{tabular}

\end{solution*}

\end{document}
