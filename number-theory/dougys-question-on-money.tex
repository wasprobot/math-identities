\documentclass{article}
\usepackage{amsthm,amssymb,amsmath,multirow,multicol,booktabs}
\usepackage[shortlabels]{enumitem}
\usepackage[utf8]{inputenc}
\usepackage[english]{babel}

\newtheorem*{problem*}{Problem}
\newtheorem*{solution*}{Solution}
\renewcommand{\theenumi}{\alph{enumi}\)}

\title{Dougy's question on money}
\date{7/16/2025}

\begin{document}
\maketitle

\begin{problem*}
    Doug asked a very interesting question: \textbf{How many coins 
    do we need to carry if we were to be able to pay 
    any denomination from 0 to 99 cents?}\\

    I took the liberty to change his question into something 
    completely different: \textbf{Given we have enough coins of each kind
    (Quarters, Dimes, Nickles, and Pennies) in how many different ways 
    can we represent money from 1 to 100 cents?}\\

    So here we will examine my question, and ignore his.
\end{problem*}

\begin{solution*}
    We introduce a function, $C(100, Q, D, N, P)$,
    representing the number of such combinations,
    using quarters (Q), dimes (D), nickles (N), and pennies (P)\\

    We make some simple observations. One is that $C(0,\cdots)=1$: 
    There is one combination to pay 0 cents (that is to use no coins!)\\
    
    Also $C(100,Q)=1$ (use 4 quarters), and so on\dots\\

    As we begin to dive deeper, we see that $C(5, N, P)=2$, the
    two combinations that can be visualized with the table:\\

    \begin{tabular}{|c|c|}
        \hline
        N & P \\
        \hline
        0 & 5\\
        \hline
        1 & 0\\
        \hline
    \end{tabular}

    \newpage
    And what if we were to represent 100 cents, 
    using only Nickles and Pennies? $C(100, N, P)=21$, because:\\

    \begin{tabular}{|c|c|}
        \hline
        N & P \\
        \hline
        0 & 100\\
        \hline
        1 & 95\\
        \hline
        2 & 90\\
        \hline
        \dots & \dots\\
        \hline
        20 & 0\\
        \hline
    \end{tabular}\\\\  

    Now let's explore $C(100, D, N, P)$:\\

    \begin{tabular}{|c|c|c|c|}
        \toprule
        D & N & P & Notice\\
        \hline
        \textbf{0} & 0 & 100 & \multirow{5}{*}{{
            $C(100, N, P)=21$ combinations
        }} \\
        \cline{1-3}
        0 & 1 & 95 & \\
        \cline{1-3}
        0 & 2 & 90 & \\
        \cline{1-3}
        0 & \dots & \dots & \\
        \cline{1-3}
        0 & 20 & 0 & \\

        \bottomrule
        \textbf{1} & 0 & 90 & \multirow{3}{*}{{
            $C(90, N, P)$ combinations
        }} \\
        \cline{1-3}
        1 & 1 & 85 & \\
        \cline{1-3}
        1 & \dots & \dots & \\

        \bottomrule
        \textbf{2} & \multicolumn{3}{|c|}{{$C(80, N, P)$}} \\

        \bottomrule
        \textbf{3} & \multicolumn{3}{|c|}{{$C(70, N, P)$}} \\

        \bottomrule
        \textbf{\dots} & \multicolumn{3}{|c|}{{\dots}} \\

        \bottomrule
        \textbf{10} & \multicolumn{3}{|c|}{{$$C(0, N, P)=1$$}} \\

        \bottomrule
    \end{tabular}\\\\

    Hence we that $C(100, D, N, P)$ depends on various 
    $C(100, N, P)$, $C(90, N, P)$, etc.
    Similarly, it's easy to see that:\\

    \begin{tabular}{|c|c|c|c|}
        \hline
        Q & D & N & P \\
        \bottomrule
        \textbf{0} & \multicolumn{3}{|c|}{{$$C(100, D, N, P)$$}} \\

        \bottomrule
        \textbf{1} & \multicolumn{3}{|c|}{{$$C(75, D, N, P)$$}} \\

        \bottomrule
        \textbf{2} & \multicolumn{3}{|c|}{{$$C(50, D, N, P)$$}} \\

        \bottomrule
        \textbf{3} & \multicolumn{3}{|c|}{{$$C(25, D, N, P)$$}} \\

        \bottomrule
        \textbf{4} & \multicolumn{3}{|c|}{{$$C(0, D, N, P)=1$$}} \\

        \bottomrule
    \end{tabular}\\\\  

\end{solution*}

\end{document}
