\documentclass{article}
\usepackage{amsthm,amssymb}
\usepackage[shortlabels]{enumitem}
\usepackage[utf8]{inputenc}
\usepackage[english]{babel}

\newtheorem*{problem*}{Problem}
\newtheorem*{solution*}{Solution}
\renewcommand{\theenumi}{\alph{enumi}\)}

\title{Euler's solution to the Basel Problem}
\date{3/4/2020}

\begin{document}
\maketitle
 
\begin{problem*}
    Find the infinite sum
    $1+\frac{1}{4}+\frac{1}{9}+\frac{1}{16}+\dots$.
\end{problem*}

\begin{solution*}
    To solve this problem Euler first factorizes a quadratic, $P(x)$ 
    with $P(0)=1$ and solutions $x=a$ and $x=b$ as:\\
    
    $P(x)=(1-\frac{x}{a})(1-\frac{x}{b})$\\
    
    Similarly, a third-degree polynomial, $P(x)$ 
    with $P(0)=1$ and solutions $x=a,x=b$ and $x=c$ 
    can be factorized as:\\
    
    $P(x)=(1-\frac{x}{a})(1-\frac{x}{b})(1-\frac{x}{c})$\\

    In general, an infinite-degree polynomial, $P(x)$ 
    with $P(0)=1$ and solutions $x=a,x=b,x=c\dots$
    can be factorized as:\\
    
    $P(x)=(1-\frac{x}{a})(1-\frac{x}{b})(1-\frac{x}{c})\dots$\\

    Then he considers an infinite-degree polynomial:\\

    $P(x) = 1-\frac{x^2}{3!}+\frac{x^4}{5!}-\frac{x^6}{7!}\dots$\\
    
    Since $P(0)=1$, to find the solutions to $P(x)$:\\

    $0 = 1-\frac{x^2}{3!}+\frac{x^4}{5!}-\frac{x^6}{7!}\dots$
    or $0 = \frac{x(1-\frac{x^2}{3!}+\frac{x^4}{5!}-\frac{x^6}{7!}\dots)}{x}$\\
    
    $\therefore \frac{sin(x)}{x} = 0$, 
    using Newton's expansion of $sin(x)$.\\

    To solve for $x$, we consider all cases where $sin(x)=0$
    i.e., $x=0, \pm\pi, \pm2\pi, \pm3\pi\dots$\\

    Omitting $x=0$ since that will render the fraction 
    $\frac{sin(x)}{x}$ undefined, we obtain that $x=\pm\pi, \pm2\pi, \pm3\pi\dots$
    are solutions to the infinite-degree polynomial, $P(x)$. 
    Thus the polynomial can be factorized as follows:\\

    $P(x) = (1-\frac{x}{\pi})(1-\frac{x}{-\pi})
    (1-\frac{x}{2\pi})(1-\frac{x}{-2\pi})
    (1-\frac{x}{3\pi})(1-\frac{x}{-3\pi})\dots$\\

    $= (1-\frac{x}{\pi})(1+\frac{x}{\pi})
    (1-\frac{x}{2\pi})(1+\frac{x}{2\pi})
    (1-\frac{x}{3\pi})(1+\frac{x}{3\pi})\dots$\\

    Multilying the terms in pairs, we obtain:\\
    
    $P(x) = [1-\frac{x^2}{(\pi)^2}]
    [1-\frac{x^2}{(2\pi)^2}]
    [1-\frac{x^2}{(3\pi)^2}]\dots$\\
    
    $= [1-\frac{x^2}{\pi^2}]
    [1-\frac{x^2}{4\pi^2}]
    [1-\frac{x^2}{9\pi^2}]\dots$\\

    Multilying the terms out for the first two terms::\\
    
    $P(x) = 1 + x^2 [
        -\frac{1}{\pi^2}
        -\frac{1}{4\pi^2}
        -\frac{1}{9\pi^2}\dots]
    $ + higher terms of $x$\\
    
    $= 1 - x^2 [
        \frac{1}{\pi^2}
        +\frac{1}{4\pi^2}
        +\frac{1}{9\pi^2}\dots]
    $ + higher terms of $x$\\
    
    Equating to the definition of $P(x)$ 
    and comparing coefficients of $x^2$,\\

    $\frac{1}{3!} = 
        \frac{1}{\pi^2}
        +\frac{1}{4\pi^2}
        +\frac{1}{9\pi^2}\dots$\\
        
    $\frac{1}{3!} = 
        \frac{1}{\pi^2}(1
        +\frac{1}{4}
        +\frac{1}{9}
        +\frac{1}{16}\dots
        )$\\
    
    Therefore the original sum,\\

    $S=1
    +\frac{1}{4}
    +\frac{1}{9}
    +\frac{1}{16}\dots
    =\frac{\pi^2}{6}$.
\end{solution*}

\end{document}
