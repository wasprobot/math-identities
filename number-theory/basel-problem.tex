\documentclass{article}
\usepackage{amsthm,amssymb,amsmath}
\usepackage[shortlabels]{enumitem}
\usepackage[utf8]{inputenc}
\usepackage[english]{babel}

\newtheorem*{problem*}{Problem}
\newtheorem*{solution*}{Solution}
\renewcommand{\theenumi}{\alph{enumi}\)}

\title{Euler's solution to the Basel Problem}
\date{3/4/2020}

\begin{document}
\maketitle
 
\begin{problem*}
    Find the infinite sum
    $S=1+\frac{1}{4}+\frac{1}{9}+\frac{1}{16}+\dots$.
\end{problem*}

\begin{solution*}
    To solve this problem Euler first shows
    a general way to factorize a \textbf{general polynomial of infinite-degree}
    (one that has infinitely many terms in a variable,
    with powers of the variables raised to $\infty$).
    But first let's take a \textbf{quadratic} 
    (a polynimial in degree 2) as an example and 
    try to factorize it. Suppose the quadratic $P(x)$ 
    satisfies $P(0)=1$ and has for solutions, 
    $x=a$ and $x=b$. It is easy to see that $P(x)$
    can be factorized as:\\
    
    $P(x)=(1-\frac{x}{a})(1-\frac{x}{b})$\\
    
    Similarly, a \textbf{third-degree polynomial}, $P(x)$ 
    with $P(0)=1$ and solutions $x=a,x=b$ and $x=c$ 
    can be factorized as:\\
    
    $P(x)=(1-\frac{x}{a})(1-\frac{x}{b})(1-\frac{x}{c})$\\

    Now he chooses a \textbf{general polynomial of infinite-degree}, 
    $P(x)$ with $P(0)=1$ and solutions $x=a,x=b,x=c\dots$
    and extends the same factorization as:\\
    
    $P(x)=(1-\frac{x}{a})(1-\frac{x}{b})(1-\frac{x}{c})\dots$\\

    Keeping that factorization technique in mind,
    he then considers an infinite-degree polynomial:\\

    \begin{equation}
        P(x) = 1-\frac{x^2}{3!}+\frac{x^4}{5!}-\frac{x^6}{7!}\dots
    \end{equation}\\


    To solve for $x$, a common technique is to equate 
    $P(x)$ to zero. Therefore:\\
    
    \begin{align*}
        P(x) &= 1-\frac{x^2}{3!}+\frac{x^4}{5!}-\frac{x^6}{7!}\dots &= 0\\
        & \frac{x(1-\frac{x^2}{3!}+\frac{x^4}{5!}-\frac{x^6}{7!}\dots)}{x} &= 0
        && \text{(multiplying both numerator and denominator by } x)\\
        & \frac{x-\frac{x^3}{3!}+\frac{x^5}{5!}-\frac{x^7}{7!}\dots}{x} &= 0\\
        \therefore P(x) &= \frac{\sin(x)}{x} &= 0
        && \text{(using Newton's expansion of sine)}\\
    \end{align*}

    \textbf{Newton's expansion of sine} states that:
    $sin(x) = \frac{x-\frac{x^3}{3!}+\frac{x^5}{5!}-\frac{x^7}{7!}\dots}{x}$.\\
    
    To solve for $x$, we consider all cases where $sin(x)=0$
    i.e., $x=0, \pm\pi, \pm2\pi, \pm3\pi\dots$\\

    Omitting $x=0$ as that will render the fraction 
    $\frac{sin(x)}{x}$ undefined, we obtain that $x=\pm\pi, \pm2\pi, \pm3\pi\dots$
    are solutions to the infinite-degree polynomial, $P(x)$.\\

    Further, since $P(0)=1-0+0-0+\dots=1$, 
    it falls in the category of the \textbf{general polynomials of infinite-degree} above.
    Therefore the polynomial can be factorized as follows:\\

    \begin{align*}
        P(x) &= (1-\frac{x}{\pi})(1-\frac{x}{-\pi})
        (1-\frac{x}{2\pi})(1-\frac{x}{-2\pi})
        (1-\frac{x}{3\pi})(1-\frac{x}{-3\pi})\dots\\
        &= (1-\frac{x}{\pi})(1+\frac{x}{\pi})
        (1-\frac{x}{2\pi})(1+\frac{x}{2\pi})
        (1-\frac{x}{3\pi})(1+\frac{x}{3\pi})\dots\\
        &= [1-\frac{x^2}{\pi^2}]
        [1-\frac{x^2}{4\pi^2}]
        [1-\frac{x^2}{9\pi^2}]\dots
        &&\text{(by multilying the terms in pairs)}\\
    \end{align*}    

    Multilying the terms out for the first two degrees of $x$:\\

    \begin{align*}        
        P(x) &= 1 + x^2 [
            -\frac{1}{\pi^2}
            -\frac{1}{4\pi^2}
            -\frac{1}{9\pi^2}\dots]
        &&\text{+ terms with higher degrees of $x$}\\
        &= 1 - x^2 [
            \frac{1}{\pi^2}
            +\frac{1}{4\pi^2}
            +\frac{1}{9\pi^2}\dots]
        &&\text{+ terms with higher degrees of $x$}\\
    \end{align*}
    
    Equating to the definition of $P(x)$ from (1) above,\\

    \begin{align*}        
        P(x) &= 1-\frac{x^2}{3!}+\frac{x^4}{5!}-\frac{x^6}{7!}\dots\\
            &= 1 - x^2 [
                \frac{1}{\pi^2}
                +\frac{1}{4\pi^2}
                +\frac{1}{9\pi^2}\dots]
            &&\text{+ terms with higher degrees of $x$}\\
    \end{align*}

    \begin{align*}        
        \frac{1}{3!} &= 
            \frac{1}{\pi^2}
            +\frac{1}{4\pi^2}
            +\frac{1}{9\pi^2}\dots
        &&\text{(by equating the coefficients of $x^2$)}\\
        \frac{1}{3!} &= 
            \frac{1}{\pi^2}(1
            +\frac{1}{4}
            +\frac{1}{9}
            +\frac{1}{16}\dots
            )\\
    \end{align*}
    
    Therefore the original sum,
    \begin{center}
        \boxed{
            S=1
            +\frac{1}{4}
            +\frac{1}{9}
            +\frac{1}{16}\dots
            =\frac{\pi^2}{6}
            }
    \end{center}
\end{solution*}

\end{document}
