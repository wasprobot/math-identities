\documentclass{article}
\usepackage[utf8]{inputenc}
\usepackage[english]{babel}
\usepackage{amsmath,amssymb}
\usepackage{mathtools}
\usepackage{amsfonts}
\usepackage{amssymb}
\usepackage{tikz}
\usepackage{amsthm}

\newtheorem*{conjecture}{Conjecture}
\newtheorem*{corollary}{Corollary}
\newtheorem*{proposition}{Proposition}

\title{$\mathbb{R}, \mathbb{Q}, \mathbb{N}, \mathbb{Z}$}

\begin{document}

\maketitle

\begin{conjecture}The sum and product of two rational numbers
is also  rational number.
\end{conjecture}
\begin{proof}If $a=\frac{a_1}{a_2}, b=\frac{b_1}{b_2} \in \mathbb{Q},
    a_2 \ne{0}, b_2 \ne{0}$,\\
    
    $a+b=\frac{a_1b_2+b_2a_1}{a_2b_2}$. Since $a_2b_2 \ne{0}, (a_2b_2), (a_1b_2+b_2a_1) \in \mathbb{N}$,
    $a+b$ is rational.\\

    $ab=\frac{a_1b_1}{a_2b_2}$. Since $a_2b_2 \ne{0}, (a_1b_1), (a_2b_2) \in \mathbb{N}$,
    $ab$ is rational.\\
\end{proof}
\begin{conjecture}If $a>b$ and $b>c$ then $a>c$.
\end{conjecture}
\begin{proof}Let $\mathbb{P}$ be the set of positive real numbers.
    $a>b,b>c \implies (a-b),(b-c) \in \mathbb{P}$. 
    \textbf{The Ordered Properties of $\mathbb{R}$} hold that $a,b\in\mathbb{P}\implies(a+b)\in\mathbb{P}$.\\

    Therefore from the hypothesis $(a-b)+(b-c)=(a-c)\in\mathbb{P}$.
    In other words, $a>c$.
\end{proof}
% \begin{corollary}
%     If $p \in \mathbb{N}$ is prime and $p \ge 7$
%     then $p$ has a multiple whose digits are $p-1$ $1$'s.
% \end{corollary}
    
% We define \textit{Repunit}, a natural number
whose digits are all $1$'s (e.g., $1, 11, 111 \dots$).
$r_n$ is a repunit with $n$ digits:

\begin{equation*}
r_n = 10^0 + 10^1 + 10^2 \dots + 10^{n-1}
= \sum_{k=0}^{n-1}{10^k}
\end{equation*}
    
\end{document}
