\documentclass{article}
\usepackage[utf8]{inputenc}
\usepackage[english]{babel}
\usepackage{amsmath,amssymb}
\usepackage{mathtools}
\usepackage{amsfonts}
\usepackage{amssymb}
\usepackage{tikz}
\usepackage{amsthm}

\newtheorem*{conjecture}{Conjecture}
\newtheorem*{corollary}{Corollary}
\newtheorem*{proposition}{Proposition}

\title{Real roots of Natural Numbers}

\begin{document}

\maketitle

\begin{conjecture}A real root of a natural number is either
    an integer or irrational.
\end{conjecture}
\begin{proof}We begin by observing that if a prime number $y$ 
    divides any power $x^p$, it also divides 
    $x$ because otherwise it would have to be ``composed" of other
    factors of $x$, which is impossible for a prime.\\

    Now let $n,k\in\mathbb{N}$ with $k\ge{2}$. We will prove 
    that $n^{1/k}$ is either an \textbf{integer} 
    or an \textbf{irrational number}. The only other possibility
    for a real root being a \textbf{strictly rational} number
    (of the form $p/q: p,q\in \mathbb{N}$ with $gcd(p,q) = 1$).\\

    $n=1$ is trivial. For $n>1$ let's assume that 
    $n^{1/k}$ is \textbf{strictly rational}. I.e.,
    
    $$n^{1/k} = \frac{p}{q}$$

    This implies that $p^k = nq^k$ or that 
    \textbf{$n$ divides $p^k$}.\\

    \textbf{If $n$ is prime}
    this also implies that $n$ divides $p$ (from the first observation we made)
    or that $p=np'$ for some factor $p'$.
    Using this and the fact that $k\ge{2}$, the last
    equation can be written as $nq^k = n^2{p'}^2 p^{k-2}$
    or $q^k = n{p'}^2 p^{k-2}$, i.e. $n$ divides $q^k$. 
    But we know this means $n$ divides $q$. Hence we see that
    $p$ and $q$ have a common divisor $n(>1)$. This is a contradiction
    to the definition of \textbf{strictly rational} numbers:
    $gcd(p,q)=1$.\\

    \textbf{If $n$ is not prime}
    let $n_a,n_b,n_c\dots$ be all the prime factors of $n$
    and $n' = n_a \cdot n_b \cdot n_c\dots$ be their product.
    Since $n$ divides $p^k$ so does $n'$ and so do each of $n_a,n_b,n_c\dots$.
    Again, from the observation above each of $n_a,n_b,n_c\dots$
    also divide $p$, and so does $n'$. In the same way as above
    we can show that now $p$ and $q$ share a common divisor
    $n'(>1)$ which is, again, contrary to the assumption above.\\

    In this way we see that in no case can $n^{1/k}$ be
    \textbf{strictly rational}. 
    So it must be either an integer or an irrational number.
\end{proof}
% \begin{corollary}
%     If $p \in \mathbb{N}$ is prime and $p \ge 7$
%     then $p$ has a multiple whose digits are $p-1$ $1$'s.
% \end{corollary}
    
% We define \textit{Repunit}, a natural number
whose digits are all $1$'s (e.g., $1, 11, 111 \dots$).
$r_n$ is a repunit with $n$ digits:

\begin{equation*}
r_n = 10^0 + 10^1 + 10^2 \dots + 10^{n-1}
= \sum_{k=0}^{n-1}{10^k}
\end{equation*}
    
\end{document}
