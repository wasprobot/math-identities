\documentclass{article}
\usepackage[utf8]{inputenc}
\usepackage[english]{babel}
\usepackage{amsmath,amssymb}
\usepackage{mathtools}
\usepackage{amsfonts}
\usepackage{amssymb}
\usepackage{tikz}
\usepackage{amsthm}

\newtheorem*{conjecture}{Conjecture}
\newtheorem*{corollary}{Corollary}
\newtheorem*{proposition}{Proposition}

\title{Roots – rational or not}

\begin{document}

\maketitle

\begin{conjecture}A natural root of a natural number is either
    an integer or irrational.
\end{conjecture}
\begin{proof}Let's assume $a^{1/b}$ is rational, i.e.,
    $$a^{1/b} = x/y$$
    Raising both sides to the power $b$,
    \begin{align*}
        & a = x^b/y^b\\
        \implies & x^b = a y^b\\
    \end{align*}
    
    $x^2$ has pairs of prime factors
    (each contributing to the individual $x$). Therefore
    if $x^2$ is even, there must be another $2$ as its factor.
    Hence $x$ must be even also.\\

    All of $x$'s prime factors get "tripled" when when
    calculate $x^3$. Therefore if $x^3$ is a multiple of $3$, 
    it must be a factor of $x$ as well.\\

    \textbf{If $b$ is prime}: Let all of $x$'s prime factors 
    be $\{p_1,p_2,\dots,p_x\}$. If $x^b$ is a multiple of $b$, 
    it must be a factor of $x$ as well.\\

\end{proof}
% \begin{corollary}
%     If $p \in \mathbb{N}$ is prime and $p \ge 7$
%     then $p$ has a multiple whose digits are $p-1$ $1$'s.
% \end{corollary}
    
% We define \textit{Repunit}, a natural number
whose digits are all $1$'s (e.g., $1, 11, 111 \dots$).
$r_n$ is a repunit with $n$ digits:

\begin{equation*}
r_n = 10^0 + 10^1 + 10^2 \dots + 10^{n-1}
= \sum_{k=0}^{n-1}{10^k}
\end{equation*}
    
\end{document}
